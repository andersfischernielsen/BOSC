\section{Implementation}

\subsection{Multitrådet sum}
TODO: skriv noget om CPU-delte CPU-caches.
\subsection{Multitrådet FIFO buffer som kædet liste}
\label{list.c}

\subsection{Producer-Consumer med bounded buffer}
Producer/Consumer implementationen er baseret på en buffer implementeret som en kædet liste (se sektion \ref{list.c}: \nameref{list.c}). Idéen er at en række tråde skal kunne oprette/producere elementer, mens en anden række tråde samtidigt skal kunne læse/konsumere disse elementer. Dog må der på et hvilket som helst i programmets kørsel ikke være produceret flere elementer end angivet med et kommandolinjeargument. For at løse dette problem, er der brugt 2 pthread-semaforer, som tæller op og ned, alt efter hvor mange elementer der er i bufferen, eller hvor mange der kan tilføjes til bufferen før den er fuld. Disse to semaforer \texttt{full} og \texttt{empty} er i hovedtrådens main-metode sat op til henholdsvis at have startværdierne 0 og bufferens størrelse. \texttt{full} bruges til at angive antallet af nuværende elementer i bufferen. Hvis denne er 0 og en consumer forsøger at tilgå bufferen, vil denne consumer blive nødt til at vente indtil en producer har tilføjet et element til bufferen. Hvis \texttt{empty} er 0 er bufferen fuld, og producere vil blive nødt til at vente indtil en consumer har fjernet et element.\\

\todo{Tænk over om det kan gøres snedigere og uden disse structs}Producere og consumere får tildelt deres arbejde ved hjælp af to typer structs. Producerens struct indeholder information omkring trådens id, hvor den skal starte med at producere fra, samt hvortil. Consumerens struct indeholder også et id, men derudover kun information omkring hvor mange elementer den skal konsumere før den skal stoppe.

\subsection{Banker's algorithm til håndtering af deadlock}

