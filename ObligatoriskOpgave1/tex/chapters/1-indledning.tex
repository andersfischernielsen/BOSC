\section{Indledning}
Denne rapport har til formål at beskrive implementationen af en shell til Linux skrevet i C. Implementationen skal give brugeren funktionalitet svarende til en begrænset version af Bash\footnote{\url{https://www.gnu.org/software/bash/}}. \\

Shellen skal understøtte følgende funktionalitet:
\begin{itemize}
	\item Implementationen skal fungere uafhængigt og må ikke starte Bash som en del af implementationen.
	\item Shellen skal vise hostname, hver gang der kan modtages kommandoer.
	\item Brugeren skal kunne køre enkeltstående kommandoer der starter programmer som ligger på operativsystemet, inklusiv installerede programmer.
	\item Hvis en kommando ikke eksisterer skal meddelelsen "Command not found" printes.
	\item Brugeren skal kunne starte processer i baggrunden ved at efterfølge sit input med '\&'.
	\item Brugeren skal kunne pipe kommandoer sammen ved hjælp af tegnet '|'.
	\item Brugeren skal kunne redirecte input til den første kommando til at læse fra en fil.
	\item Brugeren skal kunne redirecte output fra den sidste kommando til at skrive til en fil.
	\item Kommandoen 'exit' skal afslutte shellen.
	\item Ctrl-C skal afslutte programmet der kører i shellen, men ikke shellen selv.
\end{itemize}

\subsection{Ordbog}
\begin{itemize}
	\item Shellen vil i løbet af rapporten blive refereret til som BOSH.
	\item En proces' normale input fra terminalen kaldes standard in.
	\item En branch henviser til en gren af programmets kørsel.
	\item En pointer er en henvisning til starten af et hukommelsesområde. I C gives pointere en type, så det altid vides hvilken slags data der ligger i hukommelsesområdet.
	\item En fildeskriptor er operativsystems måde at henvise til en strøm af data. Fildeskriptorer benyttes fx til skrivning og læsning af filer og pipes.
	\item En pipe er et "rør" der kan bruges til interproceskommunikation. Pipes oprettes og opretholdes af operativsystemet. I brugerprogrammer er en pipe beskrevet som et par af fildeskriptorer, der henholdsvis referere til læse- og skriveenden af "røret".

\end{itemize}


\subsection{Start og udførsel af BOSH}
BOSH implementation er skrevet i filen \texttt{bosh.c}, i metoderne:
\begin{itemize}
	\item \texttt{char *gethostname(char *hostname)}
	\item \texttt{Cmd *reverse(Cmd* root)}
	\item \texttt{void start\_child(Cmd *command, int readPipe, int writePipe)}
	\item \texttt{void set\_in\_out(Shellcmd *shellcmd, Cmd *current\_cmd, int pipe\_ends[], int pipe\_in, int first, int *in, int *out)}
	\item \texttt{int executeshellcmd (Shellcmd *shellcmd)}
\end{itemize}

Projektet inkluderer også en Make fil som tillader brugeren at kompilere og fjerne de kompilerede filer ved henholdsvis kaldene \texttt{make} og \texttt{make clean}. Disse kald skal laves fra terminalen, eller BOSH, mens terminalens aktuelle filsti er den samme som filen \texttt{Makefile}.\\

BOSH køres ved kommandoen \texttt{./bosh} efter programmet er kompileret, mens terminalens aktuelle filsti er den samme som den eksekverbare fil \texttt{bosh}.