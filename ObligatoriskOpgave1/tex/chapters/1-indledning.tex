\section{Indledning}
Denne rapport har til formål at beskrive implementationen af en shell til Linux skrevet i C. Implementationen skal give brugeren funktionalitet svarende til en begrænset version af Bash\footnote{https://www.gnu.org/software/bash/}. 

Shellen skal understøtte følgende funktionalitet:
\begin{itemize}
	\item Implementationen skal fungere uafhængig og det må ikke kræve et system kald at starte shellen.
	\item Shellen skal vise hostname.
	\item Brugeren skal kunne køre enkeltstående operativ kommandoer som starter programmer der ligger på styresystemet.\todo{muligvis tilføj ls}
	\item Hvis en kommando ikke eksisterer skal der printes meddelelsen "Command not found"
	\item Brugeren skal kunne starte processor i baggrunden ved at postfixe sit input med '\&'.
	\item Brugeren skal kunne pipe kommandoer sammen ved hjælp af tegnet '|'.
	\item Brugeren skal kunne redirecte input af den første kommando til en fil.
	\item Brugeren skal kunne redirecte output af den sidste kommando til en fil.
	\item Kommandoen 'exit' skal afslutte shellen.
	\item Ctrl-C skal afslutte programmet der kører i shellen men ikke shellen selv.
\end{itemize}

Shellen vil i løbet af rapporten blive refereret til som BOSH.

BOSH implementation er skrevet i filen \verb+bosh.c+, i metoderne \verb+char *gethostname(char *hostname)+, \verb+Cmd* reverse(Cmd* root)+, \verb+void start_child(Cmd *command, int readPipe, int writePipe)+, \verb+void set_in_out(Shellcmd *shellcmd, Cmd *current_cmd, int pipe_ends[], int pipe_in, int first, int *in, int *out)+ og \verb+int executeshellcmd (Shellcmd *shellcmd)+. 

Projektet inkluderer også en Makefile som tillader brugeren at kompilerer og fjerne de kompilerede filer ved henholdsvis kaldene \verb+make+ og \verb+make clean+. Disse kald skal laves fra terminalen, eller BOSH, mens terminalens aktuelle filsti er den samme som make.h

BOSH køres ved kommandoen \verb+./bosh+ efter programmet er kompileret, mens terminalens aktuelle filsti er den samme som \verb+bosh+