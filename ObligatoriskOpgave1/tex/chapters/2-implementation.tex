\section{Implementation}
\subsection{Hostnavn}
Ved opstart af bosh bliver computerens hostname fundet. Hostname kan findes i en fil på \texttt{/proc} filsystemet som har fil-addressen \texttt{/proc/kernel/hostname}. Denne fil, som generes af operativsystemet, indeholder en linje som er computerens hostname.

Derfor kan vi for at få fat i hostname, bare læse den første linje i denne fil, og derefter lukke den igen.

Dette er gjort med systemkaldene \texttt{fopen}, \texttt{fclose} og \texttt{fscanf}. Alle disse systemkald findes i headerfilen \texttt{stdlib.h}.

\texttt{fopen} tager imod to argumenter, en filsti og en adgangstilladelse til enten at skrive eller læse. Da vi i BOSH kun skal læse fra filen beder vi kun om læsetilladelsen ved at give karakteren \texttt{r} som adgangstilladelse. Det betyder at vi kun kan læse fra filen. Når funktionen returnerer får vi en værdi tilbage af typen \texttt{FILE}, som kan bruges til at referere til filen.

\texttt{fscanf} tager så imod denne FILE, og derudover et format der skal læses, og til sidst, den (eller de) variable eller felter der skal læses til. Vi læser hele filen med formatet \texttt{\%s}, og vi læser hele resultatet ind i hukommelsen, begyndende derfra hvor \texttt{hostname} peger.

\texttt{fclose} lukker den \texttt{FILE} man giver den som argument.

\subsubsection{Mulige problemer}
Da \texttt{fscanf} ikke tager imod en maxlængde af \texttt{hostname} forestiller vi os at vores nuværende implementation af BOSH kan få problemer i tilfælde af at computeren der kører koden har et hostname der er længere end 100 karakterer, da det er den mængde hukommelse der tildeles feltet \texttt{hostname}.

\subsection{Enkeltstående kommandoer}
BOSH skal ikke reimplementere operativ programmer men derimod kalde starte dem. Dette gøres ved at forke shellen process ud i to processor ved hjælp af systemkaldet \verb+fork()+. Børne processen kan derefter ved hjælp af systemkaldet \verb+execvp()+ starte programmer der ligger i Path i operativsystemet. 

Den udleverede \verb+parser.h+ opsplitter automatisk brugerens input i kommandoer og parametre. Derfor kan BOSH videreføre augumenter til \verb+execvp+ således at programmet udføres som forventet.
\subsection{Baggrundsprocesser}
En proces kan startes i baggrunden i BOSH. Dette indebærer at BOSH starter processen, og ikke venter på at processen bliver færdig med eksekvering. 

Kontrollen gives tilbage til BOSH, og brugeren kan herefter evt. indtaste en ny kommando til eksekvering. 

Dette er helt konkret implementeret vha. systemkaldene \verb+execvp()+ og \verb+fork()+. Brugerens kommando bliver trukket ud af \verb+parser.c+'s resultat, og gives som parameter til \verb+execvp()+ sammen med eventuelle tilhørende argumenter. 

Der oprettes en børne-proces når brugeren indtaster en kommando. Denne kommando står for eksekvering af den ønskede kommande vha. \verb+execvp()+. 

Hvis brugeren har tastet \verb+&+, er den bolske værdi \verb+background+ i \verb+parser.c+'s resultat sat til true. 

Det tjekkes i \verb+bosh.c+ hvorvidt værdien er \verb+true+, og hvis dette er tilfældet, returnerer \verb+executeshellcmd+. Hvis ikke anvendes systemkaldet \verb+waitpid()+, og forældreprocessen venter på børneprocessens eksekvering. Herefter returneres.
\subsection{Pipes og redirection til og fra filer}
En process in og out kan sættes ved hjælp af \verb+dup()+ eller \verb+dup2()+ systemkaldene. Disse tager et programs standard in og out og sætter dem til den en og out som sættes i parametrene. En pipe tillader to processer at kommunikere ved at sætte den ene process' out til pipens out og den anden process' in til pipens in.

I BOSH skal vi håndterer både redirects og og pipes. Tilsammen er der 8 forskellige cases.
\begin{itemize}
	\item Enkeltstående kommando har redirect in og out
	\item Enkeltstående kommando har redirect in
	\item Første kommando, men ikke sidste, har redirect in, og skal skrive til en ny pipe.
	\item Eneste kommando har redirect out.
	\item Sidste, men ikke første, kommando har redirect out og skal læse fra pipen.
	\item Enkeltstående kommando har ingen redirects.
	\item Første, men ikke sidste, kommando skal skrive til en ny pipe.
	\item Sidste kommando, men ikke første, uden redirect men skal læse fra pipe.
	\item Mellemstående kommando, skal læse fra pipe og skrive til en ny pipe.
\end{itemize}
\subsection{Afslutning}
For at brugeren kan skrive \verb+exit+ i BOSH og dermed afslutte programmet, er en enkelt linie tilføjet til \verb+bosh.c+'s \verb+main()+-funktion. 

Det tjekkes vha. \verb+strcmp+ hvorvidt den givne linie fra brugeren ikke er lig \verb+"exit"+.

Hvis dette er tilfældet sættes \verb+terminate+ til \verb+1+, hvilket indikerer, at BOSH skal afslutte. Herefter printes \verb+"Exiting BOSH."+ og programmet afsluttes. 
\subsection{Luk kommandoer, men ikke BOSH}
For at gøre det muligt at CTRL+C'e sig ud af et givent kørende program anvendes \verb+signal()+ systemkaldet. Ved at kalde \verb+signal(SIGINT, SIG_DFL)+ når en børneproces startes i \verb+start_child()+-funktionen, sættes SIGINT (der sender \verb+interrupt+ til processen) tilbage til standard-opførsel.

Dette er nødvendigt da der i \verb+main()+-funktionen (som vi har fået udleveret) er kaldt \verb+signal(SIGINT, SIG_IGN)+, der beder processen (dvs. forældreprocessen) ignorere \verb+interrupt+, da det er ønsket kun at kunne afbryde brugerens kaldte processer. 

BOSH selv skal ikke afbryde.
