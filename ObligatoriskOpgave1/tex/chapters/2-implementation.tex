\section{Implementation}
\subsection{Hostnavn}
\subsection{Enkeltstående kommandoer}
\subsection{Baggrundsprocesser}
En proces kan startes i baggrunden i BOSC. \\ 
Dette indebærer at BOSC starter processen, og ikke venter på at processen bliver færdig med eksekvering. \\
Kontrollen gives tilbage til BOSC, og brugeren kan herefter evt. indtaste en ny kommando til eksekvering. \\
Dette er helt konkret implementeret vha. systemkaldene \verb+execvp()+ og \verb+fork()+. Brugerens kommando bliver trukket ud af \verb+parser.c+'s resultat, og gives som parameter til \verb+execvp()+ sammen med eventuelle tilhørende argumenter. \\
Der oprettes en børne-proces når brugeren indtaster en kommando. Denne kommando står for eksekvering af den ønskede kommande vha. \verb+execvp()+. \\
Hvis brugeren har tastet \verb+&+, er den bolske værdi \verb+background+ i \verb+parser.c+'s resultat sat til true. \\
Det tjekkes i \verb+bosh.c+ hvorvidt værdien er \verb+true+, og hvis dette er tilfældet, returnerer \verb+executeshellcmd+. Hvis ikke anvendes systemkaldet \verb+waitpid()+, og forældreprocessen venter på børneprocessens eksekvering. Herefter returneres.
\subsection{Redirection til og fra filer}
\subsection{Pipes}
\subsection{Afslutning}
\subsection{Luk kommandoer, men ikke BOSH}
