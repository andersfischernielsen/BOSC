\section{Implementation}
\subsection{Sidefejlshåndtering}
Når vores sidefejlshåndtering køres opdateres statistikken over sidefejl som det første. Det samlede antal sidefejl, samt antallet af sidefejl pr. side forøges med 1. Herefter læses sidens indgang i sidetabellen, for at undersøge om siden allerede er indlæst i den fysiske hukommelse, eller om den skal læses ind fra lageret.\\

Hvis siden allerede er indlæst, men kun har læseadgang, antages det at sidefejlen er opstået, fordi det kørende program vil have adgang til at skrive til siden. I dette tilfælde forøges antallet af sidefejl der er forårsaget af manglende skriveadgang. Dette tal printes ud ved afslutning af kørslen. Herefter opgraderes adgangen til siden til en skriveadgang.\\

Hvis ikke siden er i den fysiske hukommelse, men der er flere ledige rammer, bruges en kø til at fylde alle rammerne ud. Den næste ramme vælges, siden indlæses fra lageret, og sidetabellen opdateres, således at sidens indgang nu peger på den valgte ramme i den fysiske hukommelse. Undervejs opdateres en lokal tabel der gør det nemmere at finde ud af hvilken side en ramme peger på, på nuværende tidspunkt. Desuden tælles antallet af ledige rammer ned, så vi ikke tildeler den samme ramme flere gange, før hukommelsen er fyldt op.\\

Den sidste mulighed er at der skal indlæses en side fra disken, som ikke på nuværende tidspunkt er i den fysiske hukommelse, samt at der ikke er flere tilgængelige rammer. I dette tilfælde bruges en sideudskiftningsalgoritme til at vælge den ramme, og dermed side, der skal ofres til fordel for den side der nu skal indlæses. I afsnit \ref{pagereplacementalgorithms} beskrives de implementerede sideudskiftningsalgoritmer.\\

Når en offerramme er valgt, bruges den tidligere nævnte tabel til at finde ud af hvilken side der på nuværende tidspunkt er indlæst i rammen. Hvis siden har skriveadgang, skal indholdet skrives til lageret, før vi indlæser den ny side. Dette undersøges ved at tjekke det flag der angiver om en side har læse eller skriveadgang. Herefter indlæses den nye side fra disk til den fysiske hukommelse og sidetabellen opdateres, så der ikke længere er adgang til den gamle side, samt at rammen og læseflaget er sat på den nye side. Desuden opdateres den lokale tabel, så det stadig vides hvilken side en given ramme indeholder.

\subsection{Sideudskiftningsalgoritme}
\label{pagereplacementalgorithms}
\subsubsection{Tilfældig}
Implementationen\footnote{\texttt{main.c} linje: 74-77} af den tilfældige sideudskiftningsalgoritme er baseret på at udskifte en tilfældig side fra den fysiske hukkomelse. Det tilfældige index bliver udregnet med biblioteksfunktionen \texttt{lrand48} hvorefter der findes resten af heltalsdivisionen af antallet af rammer i den fysiske hukommelse.

\subsubsection{Kø}
Implementationen\footnote{\texttt{main.c} linje: 78-84} af køudgaven af sideudskiftningsalgoritmen fungerer ved, at et globalt heltal fungerer som indeks i den fysiske hukommelse. Hver gang sideudskiftningsalgoritmen bliver brugt, forøges dette indeks med ét og der findes resten af heltalsdivisionen af antallet af rammer i den fysiske hukommelse. Hermed vil først-ind-først-ud være opretholdt i køen.

\subsubsection{Brugerdefineret}
Implementationen\footnote{\texttt{main.c} linje: 85-106} af den brugerdefinerede algoritme er baseret på "Least Recently Used"-algoritmen\footnote{\href{https://en.wikipedia.org/wiki/Page_replacement_algorithm}{Wikipedia-artikel om sideudskiftningsalgoritmer: punkt 8.6}}. Hver gang der sker en sidefejl, tælles en sammenhørende værdi op med enten én, hvis der er tale om en læsesidefejl, eller to, hvis der er tale om en skrivesidefejl. Sideudskifteren itererer over siderne i den fysiske hukommelse og finder ud af hvilken side der har færrest sidefejl. Denne side bliver herefter skiftet ud. 

Den side med de færreste sidefejl må være den side, der \textit{mindst} bliver taget ind i hukommelsen. Derfor giver det mening at skifte denne ud, da den derfor bliver brugt mindre af det p.t. kørende program, og statistisk set har en lille chance for at blive taget ind igen.
