\section{Implementation}
\subsection{Sideudskiftnings algoritme}
\subsubsection{Tilfældig}
Implementationen af den tilfældige sideudskiftningsalgoritme er baseret på at udskifte en tilfældig side fra den fysiske hukkomelse. Det tilfældige index bliver udregnet med biblioteksfunktionen lrand48 hvorefter der findes resten af heltalsdivisionen af antallet af rammer i den fysiske hukommelse.

\subsubsection{Kø}
Implementationen af køudgaven af sideudskiftningsalgoritmen fungerer ved at at et globalt heltal fungerer som indeks i den fysiske hukommelse. Hver gang sideudskiftningsalgoritmen bliver brugt forøges indeks med én og der findes resten af heltalsdivisionen af antallet af rammer i den fysiske hukommelse.

\subsubsection{Brugerdefineret}
Den brugerdefinerede algoritme er baseret på "least recently used" algoritmen\footnote{\href{https://en.wikipedia.org/wiki/Page_replacement_algorithm}{WIKI Artikel om sideudskiftningsalgoritmer: punkt 8.6}}. Hver gang at der sker en side fejl, tælles en sammenhørende værdi op med én, eller to hvis der er tale om en skrivesidefejl. Sideudskifteren itererer over siderne i den fysiske hukkomelse og finder ud af hvilken side der har færrest sidefejl, og denne side bliver skiftet ud. Den side med de færreste sidefejl må være den side der mindst bliver taget ind i hukkommelsen, og derfor giver det mening at skifte denne ud da den statistisk set har lille chance for at blive taget ind igen.