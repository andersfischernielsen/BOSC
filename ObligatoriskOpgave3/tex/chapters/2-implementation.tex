\section{Implementation}
\subsection{Sidefejlshåndtering}
Når vores sidefejlshåndtering køres opdateres statistikken over sidefejl som det første. Det samlede antal sidefejl, samt antallet af sidefejl pr. side forøges med 1. Herefter læses sidens indgang i sidetabellen, for at undersøge om siden allerede er indlæst i den fysiske hukommelse, eller om den skal læses ind fra lageret.\\

Hvis siden allerede er indlæst, men kun har læseadgang, antages det at sidefejlen er opstået, fordi det kørende program vil have adgang til at skrive til siden. I dette tilfælde forøges antallet af sidefejl der er forårsaget af manglende skriveadgang. Dette tal printes ud ved afslutning af kørslen. Herefter opgraderes adgangen til siden til en skriveadgang.\\

Hvis ikke siden er i den fysiske hukommelse, men der er flere ledige rammer, bruges en kø til at fylde alle rammerne ud. Den næste ramme vælges, siden indlæses fra lageret, og sidetabellen opdateres, således at sidens indgang nu peger på den valgte ramme i den fysiske hukommelse.
\todo{Beskriv den generelle håndtering af sidefejl.}

\subsection{Sideudskiftnings algoritme}\todo{For hvert afnit, peg på hvor i koden denne algoritme ses}
\subsubsection{Tilfældig}
Implementationen af den tilfældige sideudskiftningsalgoritme er baseret på at udskifte en tilfældig side fra den fysiske hukkomelse. Det tilfældige index bliver udregnet med biblioteksfunktionen lrand48 hvorefter der findes resten af heltalsdivisionen af antallet af rammer i den fysiske hukommelse.

\subsubsection{Kø}
Implementationen af køudgaven af sideudskiftningsalgoritmen fungerer ved at at et globalt heltal fungerer som indeks i den fysiske hukommelse. Hver gang sideudskiftningsalgoritmen bliver brugt forøges indeks med én og der findes resten af heltalsdivisionen af antallet af rammer i den fysiske hukommelse.

\subsubsection{Brugerdefineret}
Den brugerdefinerede algoritme er baseret på "least-used" algoritmen\footnote{See ref}. Hver gang at der sker en side fejl, tælles en sammenhørende værdi op med én, eller to hvis der er tale om en skrivesidefejl. Sideudskifteren itererer over siderne i den fysiske hukkomelse og finder ud af hvilken side der har færrest sidefejl, og denne side bliver skiftet ud. Den side med de færreste sidefejl må være den side der mindst bliver taget ind i hukkommelsen, og derfor giver det mening at skifte denne ud da den statistisk set har lille chance for at blive taget ind igen.